---
title: "Lab - Global plastic waste"
output: 
  tufte::tufte_html:
    tufte_variant: "envisioned"
    highlight: pygments
link-citations: yes
---




Plastic pollution is a major and growing problem, negatively affecting oceans and wildlife health. [Our World in Data](https://ourworldindata.org/plastic-pollution) has a lot of great data at various levels including globally, per country, and over time. For this lab we focus on data from 2010.

Additionally, National Geographic recently ran a data visualization communication contest on plastic waste as seen [here](https://www.nationalgeographic.org/funding-opportunities/innovation-challenges/plastic/dataviz/).

# Learning goals

- Visualize numerical and categorical data.
- Recreate visualizations.
<!-- - Get more practice using with Git and GitHub. -->

# Getting started

First, open the R Markdown document `lab-02-plastic-waste.Rmd` and Knit it. Make sure it compiles without errors.

# Packages

We'll use the **tidyverse** package for this analysis. Run the following code in the Console to load this package.


```r
library(tidyverse)
library(tufte)
```

# Data

The dataset for this assignment can be found as a csv file in the `data` folder of your repository. You can read it in using the following.


```r
plastic_waste <- read_csv("data/plastic-waste.csv")
```

The variable descriptions are as follows:

- `code`: 3 Letter country code
- `entity`: Country name
- `continent`: Continent name
- `year`: Year
- `gdp_per_cap`: GDP per capita constant 2011 international $, rate
- `plastic_waste_per_cap`: Amount of plastic waste per capita in kg/day
- `mismanaged_plastic_waste_per_cap`: Amount of mismanaged plastic waste per capita in kg/day
- `mismanaged_plastic_waste`: Tonnes of mismanaged plastic waste
- `coastal_pop`: Number of individuals living on/near coast
- `total_pop`: Total population according to Gapminder

# Exercises

Let's start by taking a look at the distribution of plastic waste per capita in 2010.


```r
ggplot(data = plastic_waste, 
       aes(x = plastic_waste_per_cap)) +
  geom_histogram(binwidth = 0.2)
```

```
## Warning: Removed 51 rows containing non-finite values (stat_bin).
```

![plot of chunk plastic_waste_per_cap-hist](figure/plastic_waste_per_cap-hist-1.png)

One country stands out as an unusual observation at the top of the distribution. One way of identifying this country is to filter the data for countries where plastic waste per capita is greater than 3.5 kg/person.


```r
plastic_waste %>%
  filter(plastic_waste_per_cap > 3.5)
```

```
## # A tibble: 1 x 10
##   code  entity continent  year gdp_per_cap plastic_waste_p… mismanaged_plas…
##   <chr> <chr>  <chr>     <dbl>       <dbl>            <dbl>            <dbl>
## 1 TTO   Trini… North Am…  2010      31261.              3.6             0.19
## # … with 3 more variables: mismanaged_plastic_waste <dbl>, coastal_pop <dbl>,
## #   total_pop <dbl>
```

Did you expect this result? You might consider doing some research on Trinidad and Tobago to see why plastic waste per capita is so high there, or whether this is a data error.

1. Plot, using histograms, the distribution of plastic waste per capita faceted by continent. What can you say about how the continents compare to each other in terms of their plastic waste per capita?


```r
ggplot(data=plastic_waste)+geom_density(mapping=aes(x=plastic_waste_per_cap,color=continent))
```

```
## Warning: Removed 51 rows containing non-finite values (stat_density).
```

![plot of chunk unnamed-chunk-1](figure/unnamed-chunk-1-1.png)



Another way of visualizing numerical data is using density plots.


```r
ggplot(data = plastic_waste, 
       aes(x = plastic_waste_per_cap)) +
  geom_density()
```

```
## Warning: Removed 51 rows containing non-finite values (stat_density).
```

![plot of chunk plastic_waste_per_cap-dens](figure/plastic_waste_per_cap-dens-1.png)

And compare distributions across continents by coloring density curves by continent.


```r
ggplot(data = plastic_waste, 
       mapping = aes(x = plastic_waste_per_cap, 
                     color = continent)) +
  geom_density()
```

```
## Warning: Removed 51 rows containing non-finite values (stat_density).
```

![plot of chunk plastic_waste_per_cap-dens-color](figure/plastic_waste_per_cap-dens-color-1.png)

The resulting plot may be a little difficult to read, so let's also fill the curves in with colors as well.


```r
ggplot(data = plastic_waste, 
       mapping = aes(x = plastic_waste_per_cap, 
                     color = continent, 
                     fill = continent)) +
  geom_density()
```

```
## Warning: Removed 51 rows containing non-finite values (stat_density).
```

![plot of chunk plastic_waste_per_cap-dens-color-fill](figure/plastic_waste_per_cap-dens-color-fill-1.png)

The overlapping colors make it difficult to tell what's happening with the distributions in continents plotted first, and hence coverred by continents plotted over them. We can change the transparency level of the fill color to help with this. The `alpha` argument takes values between 0 and 1: 0 is completely transparent and 1 is completely opaque. There is no way to tell what value will work best, so you just need to try a few.


```r
ggplot(data = plastic_waste, 
       mapping = aes(x = plastic_waste_per_cap, 
                     color = continent, 
                     fill = continent)) +
  geom_density(alpha = 0.7)
```

```
## Warning: Removed 51 rows containing non-finite values (stat_density).
```

![plot of chunk plastic_waste_per_cap-dens-color-fill-alpha](figure/plastic_waste_per_cap-dens-color-fill-alpha-1.png)

This still doesn't look great...

1. Recreate the density plots above using a different (lower) alpha level that works better for displaying the density curves for all continents.


```r
ggplot(data = plastic_waste) +
  geom_density(mapping = aes(x = plastic_waste_per_cap, 
                     color = continent, 
                     fill = continent),alpha = 0.2)
```

```
## Warning: Removed 51 rows containing non-finite values (stat_density).
```

![plot of chunk unnamed-chunk-2](figure/unnamed-chunk-2-1.png)


1. Describe why we defined the `color` and `fill` of the curves by mapping aesthetics of the plot but we defined the `alpha` level as a characteristic of the plotting geom.




And yet another way to visualize this relationship is using side-by-side box plots.


```r
ggplot(data = plastic_waste, 
       mapping = aes(x = continent, 
                     y = plastic_waste_per_cap)) +
  geom_boxplot()
```

```
## Warning: Removed 51 rows containing non-finite values (stat_boxplot).
```

![plot of chunk plastic_waste_per_cap-box](figure/plastic_waste_per_cap-box-1.png)

1. Learn something new: violin plots! Read about them at http://ggplot2.tidyverse.org/reference/geom_violin.html, 
and convert your side-by-side box plots from the previous task to violin plots. 
What do the violin plots reveal that box plots do not? What features are 
apparent in the box plots but not in the violin plots?


```r
ggplot(data = plastic_waste, 
       mapping = aes(x = continent, 
                     y = plastic_waste_per_cap)) +
  geom_violin()
```

```
## Warning: Removed 51 rows containing non-finite values (stat_ydensity).
```

![plot of chunk unnamed-chunk-3](figure/unnamed-chunk-3-1.png)



```marginfigure
Remember that we use `geom_point()` to make scatterplots.
```

1. Visualize the relationship between plastic waste per capita and mismanaged plastic waste per capita using a scatterplot. Describe the relationship.


```r
ggplot(data=plastic_waste,
       mapping=aes(x=plastic_waste_per_cap, y=mismanaged_plastic_waste_per_cap)) +
  geom_point()
```

```
## Warning: Removed 51 rows containing missing values (geom_point).
```

![plot of chunk unnamed-chunk-5](figure/unnamed-chunk-5-1.png)

1. Color the points in the scatterplot by continent. Does there seem to be any clear distinctions between continents with respect to how plastic waste per capita and mismanaged plastic waste per capita are associated?


```r
ggplot(data=plastic_waste,
       mapping=aes(x=plastic_waste_per_cap, 
                   y=mismanaged_plastic_waste_per_cap,
                   color=continent)) +
  geom_point()
```

```
## Warning: Removed 51 rows containing missing values (geom_point).
```

![plot of chunk unnamed-chunk-6](figure/unnamed-chunk-6-1.png)


1. Visualize the relationship between plastic waste per capita and total population as well as plastic waste per capita and coastal population. Do either of these pairs of variables appear to be more strongly linearly associated?


```r
ggplot(data=plastic_waste,
       mapping=aes(x=total_pop,
                   y=plastic_waste_per_cap,
                   color=continent)) +
  geom_point()
```

```
## Warning: Removed 61 rows containing missing values (geom_point).
```

![plot of chunk unnamed-chunk-7](figure/unnamed-chunk-7-1.png)

```r
ggplot(data=plastic_waste,
       mapping=aes(x=coastal_pop,
                   y=plastic_waste_per_cap,
                   color=continent)) + 
  geom_point()
```

```
## Warning: Removed 51 rows containing missing values (geom_point).
```

![plot of chunk unnamed-chunk-7](figure/unnamed-chunk-7-2.png)



```marginfigure
Hint: The x-axis is a calculated variable. One country with plastic waste per capita over 3 kg/day has been filtered out. And the colors are from the viridis color palette. Take a look at the functions starting with `scale_color_viridis_*`.
```

1. Recreate the following plot, and interpret what you see in context of the data.


```r
ggplot(data=(plastic_waste %>% filter(plastic_waste_per_cap < 3.5))) + 
  geom_point(mapping=aes(x=coastal_pop/total_pop,
                   y=plastic_waste_per_cap,
                   color=continent)) + 
  geom_smooth(mapping=aes(x=coastal_pop/total_pop,
                   y=plastic_waste_per_cap))
```

```
## `geom_smooth()` using method = 'loess' and formula 'y ~ x'
```

```
## Warning: Removed 10 rows containing non-finite values (stat_smooth).
```

```
## Warning: Removed 10 rows containing missing values (geom_point).
```

![plot of chunk unnamed-chunk-9](figure/unnamed-chunk-9-1.png)


<img src="data/plastic-waste-plot.png" title="plot of chunk plot" alt="plot of chunk plot" width="100%" />
